\section*{Abstract}
Die Darstellung von Feldlinien im dreidimensionalen Raum benötigen eine hohe Datenmenge. Ziel dieser Arbeit ist es eine verlustbehaftete Kompression für Feldlinien zu entwickeln, welches eine schnelle Übertragung und effizientes Caching ermöglicht. Im Vorfeld wurde eine Kompressionsverfahren entwickelt. Die Kompressionsrate des Ansatzes war zu niedrig und verbrauchte zu viel Arbeitsspeicher für das Caching. In dieser Arbeit wurden drei Kompressionsverfahren entwickelt, welche die Effizienz der Übertragung und des Cachings verbessern. Mit einer Prädiktiven Kodierung konnte einen Kompressionsfaktor von $13.6$ zu einer vergleichbaren Qualität erreicht werden. Die Kompressionsartefakte wirken sich meist als Verschiebung der Feldlinie aus. Die Form bleibt erhalten. In einer Visualisierung sind die Artefakte dieser Kompression für das menschliche Auge schwieriger zu entdecken als die der anderen entwickelten Verfahren.

Die Prädiktive Kodierung besteht