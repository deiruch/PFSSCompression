\section*{Abstract}
Ziel dieser Arbeit ist es ein verlustbehaftetes Kompressionverfahren von wissenschaftlichen Daten zu entwickeln, welche die Übertragung und Zwischenspeicherung der Daten ermöglicht. Die Kompressionsartefakte müssen dabei in Grenzen gehalten werden, sodass das menschliche Auge keine Artefakte in der Visualisierung erkennen kann.

In dieser Arbeit wurden drei unterschiedliche Verfahren für die Kompression von wissenschaftlichen Daten entwickelt. Variante Adaptives Subsampling, DCT und Prediktive Kodierung.
Adaptives Subsampling erreicht eine Kompressionsrate von $11.6$, indem es einen grossteil der Informationen verwirft.
Die DCT Variante erreicht eine Kompressionsrate von $14.1$. Dieser Ansatz führt Artefakte ein, welche in der Visualisierung als störend empfunden werden.
Die Prediktive Kodierung erreicht eine Kompressionsrate von $12.6$. Es ist beinahe Artefaktfrei.
Die Dekompression wurde erweitert?