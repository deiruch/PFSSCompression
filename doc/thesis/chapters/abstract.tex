\section*{Abstract}
Die Darstellung von Feldlinien im dreidimensionalen Raum benötigen eine hohe Datenmenge. Ziel dieser Arbeit ist es eine verlustbehaftete Kompression für Feldlinien zu entwickeln, welches eine effiziente Übertragung und das Caching im Arbeitsspeicher ermöglicht. Im Vorfeld wurde eine Kompressionsverfahren entwickelt. Die Kompressionsrate des Ansatzes war zu niedrig und verbrauch zu viel Arbeitsspeicher für das Caching. In dieser Arbeit wurden drei Kompressionsverfahren entwickelt, welche die Effizienz der Übertragung und des Cachings verbessern. Mit einer Prädiktiven Kodierung konnte einen Kompressionsfaktor von $13.6$ zu einer vergleichbaren Qualität erreicht werden.
In dieser Arbeit wurden drei Verfahren entwickelt: eine Kompression mit Adaptivem Subsampling, eine Kompression mit einer Diskreten Kosinus Transformation und eine Kompression mit Prädiktoren. Die höchste Kompressionsrate wurde mit der Diskreteten Kosinus Transformation erreicht, während die Kompression mit Adaptiven Subsamplings minimale Kompressionsartefakte aufweist. Die Kompression mit Prädiktoren ist ein Kompromiss zwischen Kompressionsrate und Artefakte.

Die Übertragung und Zwischenspeicherung von Feldlinien ist mit den entwickelten Verfahren möglich. Der limitierende Faktor für die Kompression ist die Artefaktbildung: Ringing oder Ringing-Ähnliche Artefakte sind in Visualisierungen der Feldlinien störend. Wenn die Visualisierung ein Heranzoomen erlaubt, sind auch schwach ausgeprägte Artefakte zu erkennen. Eine höhere Kompressionsrate ist mit Verfahren zu erreichen, welche kaum oder keine Ringing Artefakte produzieren wie Wavelet Transformation, Curve Fitting oder Compressive Sensing.