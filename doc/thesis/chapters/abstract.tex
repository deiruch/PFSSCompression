\section*{Abstract}
JHelioviewer ist eine Applikation, welche Mess- und Simulationsdaten der Sonne visualisiert. Die Daten für die Visualisierung werden zur Laufzeit über eine Internetverbindung geladen. Ziel dieser Arbeit ist es eine verlustbehaftete Kompression für wissenschaftliche Simulationen zu entwickeln, welche die Übertragung und Zwischenspeicherung ermöglicht.

In dieser Arbeit wurden drei Verfahren entwickelt: Kompression mit Adaptivem Subsampling, Kompression mit einer Diskreten Kosinus Transformation und Kompression mit Prädiktoren. Die höchste Kompressionsrate wurde mit der Diskreteten Kosinus Transformation erreicht, während die Kompression mit Adaptiven Subsamplings minimale Kompressionsartefakte aufweist. Die Kompression mit Prädiktoren ist ein Kompromiss zwischen Kompressionsrate und Artefakte.

Die Übertragung und Zwischenspeicherung von wissenschaftlichen ist mit den entwickelten Verfahren möglich. Der Limitierende Faktor ist die Artefaktbildung: In der Visualisierung des JHelioviewers sind auch schwach ausgeprägte Artefakte störend. Für weitere Forschungen können Wavelet Transformation und Compressive sensing vielversprechend sein.
