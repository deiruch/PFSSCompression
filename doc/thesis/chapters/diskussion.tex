\section{Diskussion}

\subsection{Diskussion Lösungsansatz Adaptives Subsampling}
Der Lösungsansatz des adaptiven Subsamplings erreicht eine Kompressionsrate von $11.6$ gegenüber dem Ist-Zustand, indem es nur die Punkte überträgt, welche der JHelioviewers darstellt. Dadurch weisen die visualisierten Feldlinien die selbe Genauigkeit und die selben Artefakte auf, wie der Ist-Zustand. Falls aber mehr Punkte dargestellt werden sollen, müssen entweder mehr Punkte abgespeichert werden oder der JHelioviewer muss eine Interpolation durchführen.

Der JHelioviewer muss in der Lage sein $1000$ komprimierte Simulationen im Arbeitsspeicher abzulegen. Mit dieser Kompression werden durchschnittlich $85$ Megabyte an Arbeitsspeicher benötigt. Wenn von einer $10$ Megabit Internetverbindung ausgegangen wird, werden für das Herunterladen von $1000$ Simulationen $70$ Sekunden benötigt. Das ist eine deutliche Verbesserung gegenüber zum Ist-Zustand, welcher unter den selben Bedingungen $790$ Sekunden ($13$ Minuten) benötigt. Mit dieser Kompression können etwa $14$ komprimierte Simulationen pro Sekunde übertragen werden. Der Benutzer erhält eine flüssige Animation der Feldlinien, wenn pro Sekunde weniger als $14$ unterschiedliche Simulationen angezeigt werden müssen.

Ein Vorteil dieses Lösungsansatzes ist die unkomplizierte Dekompression. Da keine Rechenaufwändige Rücktransformationen verwendet werden braucht dieser Lösungsansatz $19$ Millisekunden für eine Dekompression (Siehe Abschnitt \ref{anhang:performance}). Mit einem Thread ist die Testmaschine in der Lage $50$ Simulationen pro Sekunde zu dekomprimieren. Gestaltet sich deshalb als die performanteste Dekompression.

\subsection{Diskussion Lösungsansatz DCT}
Die DCT Kompression kann eine Kompressionsrate von $14.1$ erreicht werden. Im Vergleich mit dem adaptiven Subsampling wird eine höhere Kompressionsrate erreicht, obwohl eine höhere Anzahl an Punkte übertragen werden. Die Problematik dieses Ansatzes liegt darin, dass die DCT Kompression Ringing Artefakte hinzufügt (siehe Abschnitt \ref{resultate:loesung1:ringing}). Die Artefakte können bei allen Kompressionsverfahren mit einer Kosinus Transformation auftreten, wie bei JPEG/JFIF Bilder und MP3 Audiodateien. Der JHelioviewer bietet die Möglichkeit, an die Feldlinien heranzuzoomen. Sobald Ringing Artefakte existieren ist der Benutzer in der Lage sie zu finden. Bei der Entwicklung der Kompression wurde nach Möglichkeiten gesucht, die Ringing Artefakte zu dämpfen. Die Daten sind Artefaktfrei bei etwa der Hälfte des Zoombereichs. Mit einer Glättung der Feldlinien konnten die Ringing Artefakte für alle Zoomstufen behoben werden. 

Beim Caching von $1000$ Simulationen benötigt dieser Ansatz $70$ Megabyte Arbeitsspeicher, etwa $15$ Megabyte weniger als der Lösungsansatz des adaptiven Subsamplings. Wenn von der selben $10$ Megabit Internenetverbindung ausgegangen wird, werden $56$ Sekunden benötigt um $1000$ Simulationen herunterzuladen. Pro Sekunde werden $17$ anstatt $14$ Simulationen übertragen. Wenn für die Movies auf einen artefaktfreien Zoom verzichtet wird, ist eine weitaus höhere Kompressionsrate möglich. Wenn komplett auf den Zoom verzichtet wird, ist eine Kompressionsrate von $24$ möglich.

Die bessere Kompressionsrate kommt auf Kosten der Komplexität der Dekompression. Die Laufzeit der Dekompression hängt im Wesentlichen von der Implementation der inversen DCT ab. Eine naive Implementation braucht für eine Dekompression etwa drei Sekunden. Die grösste Zeit wird in der Berechnung der Kosinusfunktion verbraucht. Bei der Implementation in dieser Arbeit werden die Funktionen über SoftReferences gecached. Der Cache passt sich somit an den Arbeitsspeicher an. Wie schnell die Dekompression ist, hängt im Wesentlichen vom freien Arbeitsspeicher ab. Im besten Fall ergibt das eine Laufzeit von $65$ Millisekunden und im schlechtesten Fall $350$. Das führt zu einem Durchsatz zwischen $3$ und $15$ Simulationen, pro Sekunde pro Thread.

Verbessertes Postprocessing, welches die Ringing Artefakte vermindert.



Ziel: Ist Zustand ersetzen, 1000 Simulationen mit möglichst wenig Arbeitsspeicher zu cachen. Movies bei 
Allgemein Zielerreichung