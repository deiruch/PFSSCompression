\section{Resultate}\label{resultate}
In diesem Abschnitt werden die Ergebnisse der Tests vorgestellt. Aufgrund dieser Resultate wurden die Lösungsansätze vom Abschnitt \ref{konzept} entwickelt. Für die PSNR-HVS-M Metrik existiert nicht immer ein Diagramm. Die Metrik wurde im Laufe der Arbeit entwickelt und wird ab Abschnitt \ref{resultate:loesung1:behandlung_ringing} ebenfalls gemessen.

\subsection{Lösungsansatz: Adaptives Subsampling} \label{resultate:loesung0}
Im Ist-Zustand führt der JHelioviewer nach der Dekompression ein adaptives Subsampling durch. Dieser Lösungsansatz führt das adaptive Subsampling vor der Datenübertragung durch und Kodiert die Daten mit Rar anstatt mit Gzip. Eine genauere Beschreibung des Ansatzes ist im Abschnitt \ref{konzept:loesung0} zu finden.
\begin{figure}[!htbp]
	\center
	\includegraphics[width=1\textwidth,keepaspectratio]{./pictures/resultate/loesung0/loesung0_0.png}
	\caption{Vergleich des Lösungsansatzes: Adaptives Subsampling zur Ist-Kompression.}
	\label{resultate:loesung0:loesung0_0}
\end{figure}
Wie im Diagramm \ref{resultate:loesung0:loesung0_0} erkennbar ist, braucht dieser Lösungsansatz deutlich weniger Speicher als die Ist-Kompression zur selben Genauigkeit. Mit diesem Ansatz wird eine Kompressionsrate von $11.6$ erreicht.\\
\begin{figure}[!htbp]
	\center
	\includegraphics[width=1\textwidth,height=8cm,keepaspectratio]{./pictures/resultate/loesung0/loesung0_artefakte.png}
	\caption{Artefakte des Lösungsansatzes Adaptives Subsampling.}
	\label{resultate:loesung0:artefakte}
\end{figure}
Die Abbildung \ref{resultate:loesung0:artefakte} zeigt die Artefakte, die bei der Kompression entstehen. Der Ist-Zustand weist die selben Artefakten auf.
\pagebreak

\subsection{Lösungansatz: Diskrete Kosinus Transformation}
In diesem Abschnitt wird der Lösungsansatz mittels Diskreter Kosinus Transformation behandelt. Es wurden verschiedene Transformationen getestet, welche eine Approximation mittels Kosinus Funktionen vereinfachen.\\
Um die Feldlinien optimal mit der DCT zu approximieren, müssen Ringing Artefakte, behandelt werden. Das Auftreten der Artefakte wird im Abschnitt \ref{resultate:loesung1:ringing} und die Behandlung im Abschnitt \ref{resultate:loesung1:behandlung_ringing} besprochen .

\subsubsection{Variante: DCT}\label{resultate:dct}
Diese Variante verwendet die Diskrete Kosinus Transformation. Es wird erforscht welche Kompressionsrate ohne zusätzliche Transformationen möglich ist.
\begin{figure}[!htbp]
	\center
	\includegraphics[width=1\textwidth,keepaspectratio]{./pictures/resultate/loesung1/loesung1-0/loesung1_0.png}
	\caption{Vergleich der DCT Kompression mit dem Lösungsansatz des Adaptiven Subsamplings}
	\label{resultate:loesung1:dct:resultate}
\end{figure}
Die Abbildung \ref{resultate:loesung1:dct:resultate} zeigt den Vergleich der DCT Kompression mit dem Lösungsansatz des Adaptiven Subsamplings (siehe \ref{resultate:loesung0}). Die Standardabweichung steigt unerwartet schnell an. Der Grund kann im Diagramm der Abbildung \ref{resultate:loesung1:dct:artefakte} entnommen werden. 
\begin{figure}[!htbp]
	\center
	\includegraphics[width=0.8\textwidth,height=8cm,keepaspectratio]{./pictures/resultate/loesung1/loesung1-0/loesung1_0_artefakte.png}
	\caption{Artefakte der DCT Dekompression anhand Beispieldaten}
	\label{resultate:loesung1:dct:artefakte}
\end{figure}
Der Start und Endpunkt der Feldlinie kann nicht richtig dargestellt werden. Dies ist ein typisches Problem der DCT: Ein Inputsignal wird in der DCT konzeptionell wiederholt. Bei der implemementierten Kosinus Transformation (siehe Abschnitt \ref{konzept:loesung1:kosinus}) wird das Signal jeweils in umgekehrter Reihenfolge wiederholt. Bei der Beispielfeldlinie aus Abbildung \ref{resultate:loesung1:dct:artefakte} führt die Wiederholung zu einer Diskontinuität im Signal. Die Diskontinuität wird in der DCT mit hochfrequenten Anteilen abgebildet, welche von der Quantisierung gelöscht werden. Das Ergebnis ist eine Verschiebung an den Stellen, wo die Diskontinuität auftritt. In diesem Fall ist es jeweils am Anfang und am Ende der Feldlinie.\\
Das Problem kann entweder durch eine andere Darstellung der Feldlinie oder durch zusätzliche Punkte am Anfang und am Ende der Linie gelöst werden. Die Variante, welche die Feldlinie mit Punkten erweitert, wird im Abschnitt ref{resultate:loesung1:dct:randbeh+byte} behandelt. Eine andere Darstellung der Feldlinie wird in den folgenden Abschnitten behandelt.

\subsubsection{Variante: Ableitung+DCT}\label{resultate:dct:ableitung_dct}
Vor der Diskreten Kosinus Transformation werden die Feldlinien abgeleitet. Durch diese Darstellung werden die Artefakte aus der Abbildung \ref{resultate:loesung1:dct:artefakte} behoben.\\
\begin{figure}[!htbp]
	\center
	\includegraphics[width=1\textwidth,keepaspectratio]{./pictures/resultate/loesung1/loesung1-1/loesung1_1.png}
	\caption{Vergleich der DCT Kompression der Ableitung mit der DCT Kompression}
	\label{resultate:loesung1:dct_ableitung:resultate}
\end{figure}
Das Diagramm der Abbildung \ref{resultate:loesung1:dct_ableitung:resultate} zeigt, dass die abgeleiteten Feldlinien besser approximiert werden können als die vorhergehende Variante. Bei einer vergleichbaren Genauigkeit wie der Ist-Zustand erreicht diese Variante eine Kompressionsrate von $14.3$. Eine Darstellung der Artefakte ist im Diagramm der Abbildung\ref{resultate:loesung1:dct:byte:artefakte} zu finden. Die Artefakte der Kompression äussern sich als Dämpfungen. Die Artefakte sind jedoch für das menschliche Auge nicht sichtbar. Die Dämpfung ist erst zu erkennen, wenn das Original zur Verfügung steht.
\begin{figure}[!htbp]
	\center
	\includegraphics[width=0.8\textwidth,height=8cm,keepaspectratio]{./pictures/resultate/loesung1/loesung1-6/artefakte.png}
	\caption{Artefakte der DCT Kompression der Ableitung}
	\label{resultate:loesung1:dct:byte:artefakte}
\end{figure} 

\subsubsection{Variante: PCA+Ableitung+DCT}
Die Feldlinien liegen im Allgemeinen in einer Ebene im dreidimensionalen Raum. Durch eine Principal Component Analysis (PCA)\cite{abdi2010principal} können die Feldlinien in ein lokales Koordinatensystem transformiert werden, in welchem der Z Kanal in den meisten Fällen nicht gebraucht wird.\\
Für die Rücktransformation ins Sonnen-Koordinatensystem werden pro Feldlinie zusätzlich sech Parameter für die neuen Koordinatenachsen und drei Parameter für die Verschiebung abgespeichert.
\begin{figure}[!htbp]
	\center
	\includegraphics[width=1\textwidth,keepaspectratio]{./pictures/resultate/loesung1/loesung1-4/loesung1_4.png}
	\caption{Vergleich der PCA DCT Kompression der Ableitung mit der DCT Kompression der Ableitung}
	\label{resultate:loesung1:dct:pca}
\end{figure}
Im Diagramm der Abbildung \ref{resultate:loesung1:dct:pca} sind die Resultate der Messung dargestellt. Die PCA konnte keine Verbesserung erbringen. Die zusätzlichen Parameter verbrauchen mehr Speicherplatz als durch die PCA gewonnen werden kann.

\subsubsection{Variante: Ableitung+DCT+Byte Kodierung} \label{resultate:loesung1:ableitung_dct_kodierung}
Ein Kanal einer Feldlinie kann mit 5-20 DCT-Koeffizienten ausreichend approximiert werden. Die quantisieren Koeffizienten kommen meistens zwischen -50 und +50 zu liegen. 8 Bit Genauigkeit reichen im Allgemeinen aus, um einen quantisierten Koeffizienten abzuspeichern. Um die Kompressionsrate zu verbessern, werden zwei Byte-Kodierungen eingeführt: die Längenkodierung und die Byte-Kodierung (beschrieben im Abschnitt \ref{konzept:loesung1:kodierung}).\\
\begin{figure}[!htbp]
	\center
	\includegraphics[width=1\textwidth,keepaspectratio]{./pictures/resultate/loesung1/loesung1-6/loesung1_6.png}
	\caption{Vergleich der Kompression mit und ohne Byte-Kodierung}
	\label{resultate:loesung1:dct:kodierung}
\end{figure}
Das Diagramm der Abbildung \ref{resultate:loesung1:dct:kodierung} zeigt den Einfluss der Byte-Kodierung auf die Kompressionsrate. Es bewirkt eine deutliche Verbesserung, ohne die Qualität der Kompression negativ zu beeinflussen. Bei einer Vergleichbaren Genauigkeit wie die Ist-Lösung weist diese Variante eine Kompressionsrate von $24.5$ auf.

\subsubsection{Variante: Randbehandlung+DCT+Byte Kodierung} \label{resultate:loesung1:dct:randbeh+byte}
Wenn  die Artefakte \ref{resultate:loesung1:dct:byte:artefakte} und \ref{resultate:loesung0:artefakte} vergleicht, fällt auf, dass die Variante \ref{resultate:dct} die Feldlinie genauer approximiert, falls die Ränder der Feldlinie besser dargestellt werden könnte. Wenn die Feldlinie an den Rändern mit Punkten erweitert wird, welche die Transformation vereinfachen, könnte eine bessere Kompression erreicht werden.\\
Jeder Kanal einer Feldlinie wird so erweitert, dass der Anfang und das Ende abflacht. Die Byte-Kodierung wurde aus der vorhergehenden Variante übernommen.
\begin{figure}[!htbp]
	\center	\includegraphics[width=1\textwidth,keepaspectratio]{./pictures/resultate/loesung1/loesung1-7/loesung1_7.png}
	\caption{Vergleich des Einflusses der Randbehandlung}
	\label{resultate:loesung1:dct:randbehandlung}
\end{figure}
Das Diagramm der Abbildung \ref{resultate:loesung1:dct:randbehandlung} zeigt den Vergleich der Variante mit Randbehandlung und der Variante der abgeleiteten Feldlinie (beschrieben im Abschnitt \ref{resultate:loesung1:ableitung_dct_kodierung}. Es ist zu erkennen, dass dank der Randbehandlung die Feldlinien mit weniger Bytes ähnlich genau approximiert werden können.\\
Diese Variante führt aber Artefakte ein, welche die Standardabweichung nicht erkennen kann. Im Folgenden Abschnitt \ref{resultate:loesung1:ringing} werden diese besprochen.

\subsubsection{Ringing Artefakte}\label{resultate:loesung1:ringing}
Obwhol die Variante \ref{resultate:loesung1:dct:randbeh+byte } eine vergleichbare Genauigkeit aufweist, wie die Ist-Lösung, sind auf der JHelioviewer Visualisierung deutliche Artefakte zu sehen. Die Abbildung \ref{resultate:loesung1:dct:randbehandlung:jvhartefakte} vergleicht die originalen mit dekomprimierten Feldlinien. Die Artefakte zeigen sich als Oszillationen.
\begin{figure}[!htbp]
	\center
	\frame{
	\includegraphics[width=0.8\textwidth,height=6cm,keepaspectratio]{./pictures/resultate/loesung1/ringing/actual.png}}
		\frame{
	\includegraphics[width=0.8\textwidth,height=6cm,keepaspectratio]{./pictures/resultate/loesung1/ringing/sol7.png}}
	\caption{Artefakte der Kompression. Links sind die originalen Feldlinien, rechts die Dekomprimierten.}
	\label{resultate:loesung1:dct:randbehandlung:jvhartefakte}
\end{figure} 
In diesem Fall scheint die Standardabweichung als Fehlermass zu versagen: Da die Oszillationen nahe an der originalen Feldlinie liegen, bleiben die Abstände klein. Jedoch sind die Artefakte für das menschliche Auge inakzeptabel.\\
Interessant ist, dass die die Variante der Ableitung (Abschnitt \ref{resultate:loesung1:ableitung_dct_kodierung}) ähnliche Artefakte aufweist. Im Diagramm der Abbildung \ref{resultate:loesung1:dct:byte:artefakte}, welches die Artefakte anhand einer Beispiellinie zeigt, sind keine Oszillationen zu entdecken. In der JHelioviewer Visualisierung jedoch, sind auch bei dieser Variante deutliche Oszillationen zu erkennen. Die Abbildung \ref{resultate:loesung1:dct:randbehandlung:jvhartefakte_loesung6} zeigt die Artefakte. Es ist anzumerken, dass die Artefakte weniger ausgeprägt sind, aber dennoch störend für das menschliche Auge.\\
\begin{figure}[!htbp]
	\center
	\frame{
	\includegraphics[width=0.8\textwidth,height=6cm,keepaspectratio]{./pictures/resultate/loesung1/ringing/actual.png}}
		\frame{
	\includegraphics[width=0.8\textwidth,height=6cm,keepaspectratio]{./pictures/resultate/loesung1/ringing/sol6.png}}
	\caption{Artefakte der Kompression, links sind die originalen Feldlinien, rechts die Dekomprimierten der Variante \ref{resultate:loesung1:ableitung_dct_kodierung}.}
	\label{resultate:loesung1:dct:randbehandlung:jvhartefakte_loesung6}
\end{figure}
Die oszillierenden Artefakte sind typisch für eine DCT-Kompression und sind im Allgemeinen als Ringing Artefakte \cite{wiki:ringing:artefacts} bekannt: Abrupte Steigungen im Inputsignal werden in der DCT durch hochfrequente Anteile dargestellt. Durch die Quantisierung der hochfrequenten Anteile werden oszillierende Artefakte eingefügt. Die Ringing Artefakte typische Kompressionsartefakte von DCT-basierten Verfahren wie JPEG/JFIF oder MP3.\\
Die Feldlinien, welche am stärksten von den Artefakten betroffen sind, sind die ''Weltall zur Sonne'' oder ''Sonne ins Weltall'' Feldlinien. Sie verhalten sich nicht wie harmonische Halbwellen sondern steigen oft monoton, mit teils abrupten Richtungswechseln in der Nähe der Sonnenoberfläche. Die Abbildung \ref{resultate:loesung1:dct:randbehandlung:harte_richtungswechsel} zeigt ein Beispiel solcher Feldlinien. Die abrupten Wechsel führen zu abrupten Steigungen in den einzelnen Kanälen.
\begin{figure}[!htbp]
\center
\includegraphics[width=0.4\textwidth,height=6cm,keepaspectratio]{./pictures/resultate/loesung1/ringing/haar-like.png}
	\caption{Abrupte Steigungen bei Feldlinien, welche von der Sonne ins Weltall führen.}
	\label{resultate:loesung1:dct:randbehandlung:harte_richtungswechsel}
\end{figure}

\subsubsection{Behandlung der Ringing Artefakte} \label{resultate:loesung1:behandlung_ringing}
Um eine optimale Kompression der Feldlinien zu erreichen, müssen die Ringing Artefakte behandelt werden. Abschnitt \ref{resultate:loesung1:ringing} wurde erwähnt, dass nicht alle Varianten gleich starke Artefakte verursachen. Es wurde ebenfalls besprochen, dass die Feldlinien, welche von der Sonnenoberfläche zur Oberfläche führen, kaum von den Artefakten betroffen sind. In diesem Abschnitt wird deshalb erforscht, welche Variante die ''Sonne zu Sonne'' und welche die ''Weltall zur Sonne'' oder ''Sonne ins Weltall'' Feldlinien optimal approximieren kann. Für die Messung der Artefakte wird die PSNR-HVS-M Metrik aus Abschnitt \ref{testsetup:psnr} verwendet.\\
Für den Test werden insgesamt vier Varianten verglichen. Die Varianten \ref{resultate:loesung1:ableitung_dct_kodierung} und und \ref{resultate:loesung1:dct:randbeh+byte} jeweils mit und ohne PCA Transformation. Die Transformation wird hier nochmals geprüft. Die Transformation könnte die Rining Artefakte jeweils auf einen Kanal beschränken.\\
\begin{figure}[!htbp]
	\center	\includegraphics[width=1\textwidth,keepaspectratio]{./pictures/resultate/loesung1/ringing/sts.png}
	\caption{Approximation der Feldlinien ''Sonnenoberfläche zu Sonnenoberfläche''. Je höher die PSNR-HVS-M, desto besser ist die Approximation. }	\label{resultate:loesung1:dct:behandlung_ringing:sts}
\end{figure} 
\begin{figure}[!htbp]
	\center
\includegraphics[width=1\textwidth,keepaspectratio]{./pictures/resultate/loesung1/ringing/nosts.png}
	\caption{Approximation der der Feldlinien ''Sonnenoberfläche ins Weltall'' oder ''Weltall zur Sonnenoberfläche''. Je höher die PSNR-HVS-M, desto besser ist die Approximation.}	\label{resultate:loesung1:dct:behandlung_ringing:nosts}
\end{figure}
Das Diagramm der Abbildung \ref{resultate:loesung1:dct:behandlung_ringing:sts} zeigt, wie gut die Varianten die Feldlinien approximieren können, welche von der Sonnenoberfläche zur Sonnenoberfläche führen. Es ist anzumerken, dass ein PSNR-HVS-M Wert von etwa $95 dB$ bedeutet, dass die Feldlinien kaum sichtbare Artefakte enthalten. Je höher der PSNR-HVS-M Wert, desto genauer ist die Approximation. Interessant ist, dass drei Varianten ähnlich viel Speicherplatz benötigen, für eine beinahe artefaktfreie Approximation. Bei stärkerer Quantisierung treten beiden abgeleiteten Feldlinien deutlich weniger Artefakte auf. Dies deckt sich mit den Beobachtung aus dem Abschnitt \ref{resultate:loesung1:ringing}.\\
Ebenfalls interessant ist die Variante der PCA Transformation zusammen mit der Ableitung: Diese benötigt für eine beinahe artefaktfreie Approximation am wenigsten Speicherplatz, führt aber bei stärkerer Quantisierung viele Artefakte ein.

Das Diagramm der Abbildung  \ref{resultate:loesung1:dct:behandlung_ringing:nosts} zeigt, wie gut die Varianten die die andern Typen von Feldlinien approximieren können. Hier ist zu sehen, dass die abgeleiteten Feldlinien weniger Artefakte hinzufügen, jedoch weniger deutlich als in der Abbildung \ref{resultate:loesung1:dct:behandlung_ringing:sts}. Interessant ist, dass die PCA keinen messbaren Vorteil erbrachte. Die zusätzlichen Parameter, die für die Umkehrung der PCA abgespeichert werden, verbrauchen mehr Speicherplatz als man durch die Transformation gewinnt.\\
Die DCT Variante aus Abschnitt \ref{resultate:loesung1:dct:randbeh+byte} fällt ab einer PSNR-HVS-M von $90 dB$ weniger schnell ab, als die abgeleiteten Feldlinien. Bei diesem PSNR-HVS-M Wert sind aber die Artefakte bereits zu deutlich und nicht mehr akzeptabel. Aufgrund dieser Daten wurde die Variante unter Abschnitt \ref{resultate:loesung1:ableitung_dct_kodierung} ausgewählt. 

\subsubsection{Abschliessende Variante}
Für den abschliessenden Test wurde die Variante unter Abschnitt \ref{resultate:loesung1:ableitung_dct_kodierung} ausgewählt. Diese verursacht weniger starke Ringing Artefakte als die anderen Varianten. Dies liegt an der Artefaktdämpfenden Eigenschaft der Ableitung und an der auf die auf den Typ angepasste Quantisierung.
\begin{figure}[!htbp]
	\center	\includegraphics[width=1\textwidth,keepaspectratio]{./pictures/resultate/loesung1/loesung1-12/resultate.png}
	\caption{Standardabweichung der abschliessenden DCT Variante.}	\label{resultate:loesung1:dct:abschliessend:standardabweichung}
\end{figure} 
Das Diagramm der Abbildung \ref{resultate:loesung1:dct:abschliessend:standardabweichung} zeigt die Standardabweichung der abschliessenden DCT Variante. Zu einer besseren Kompression kann sie deutlich genauer Punkte ablegen und erreicht eine Kompressionsrate von $14.1$. Die Artefakte sind mit einer PSNR-HVS-M von $94.0$ ebenfalls in Grenzen gehalten. Jedoch sind wegen dem Zoom-Feature des JHelioviewers immer noch Oszillationen zu erkennen. Das Linke Bild der Abbildung \ref{resultate:loesung1:dct:final:artefakte} zeigt die Artefakte der Dekompression. Die Artefakte sind aus der Simulation, welche diese Variante am schlechtesten approximieren konnte. Es ist ebenfalls anzumerken, dass die Artefakte bei weniger hohen Zoomstufen nicht mehr zu erkennen sind.
\begin{figure}[!htbp]
	\center
	\frame{
	\includegraphics[width=0.8\textwidth,height=5cm,keepaspectratio]{./pictures/resultate/loesung1/loesung1-12/without_average_line.png}}
		\frame{
	\includegraphics[width=0.8\textwidth,height=5cm,keepaspectratio]{./pictures/resultate/loesung1/loesung1-12/with_average_line.png}}
	\caption{Die am stärksten ausgeprägten Artefakte der abschliessenden Variante. Links ohne Glättung, rechts mit Glättung}
	\label{resultate:loesung1:dct:final:artefakte}
\end{figure}
Die Verbesserung in der Genauigkeit wurde erreicht, indem die Feldlinientypen unterschiedlich quantisiert wurden. Die ''Sonne zu Sonne'' können mit wenigeren Koeffizienten approximiert werden als die anderen Typen. Würde die Simulation nur aus diesen Feldlinien bestehen, könnte eine Kompressionsrate von $18-20$ erreicht werden zu einer ähnlichen Genauigkeit. Die Schwierigkeit liegt darin die anderen Feldlinien mit möglichst wenigen Artefakten zu approximieren. Eine Kompression mit der Diskreten Kosinus Transformation wird ab einer gewissen Zoom-Stufe immer Artefakte mit sich bringen, welche das menschliche Auge erkennen kann.\\
Der JHelioviewer kann die Artefakte verschleiern mit einer Kurvenglättung: Die komprimierten Daten enthalten mehr Punkte, als der JHelioviewer darstellen könnte. Die zusätzlichen Punkte können für die Glättung der Feldlinie verwendet werden. Den Effekt der Glättung ist in der Abbildung \ref{resultate:loesung1:dct:final:artefakte} verdeutlicht. Dadurch können die Artefakte auch bei höheren Zoomstufen verschleiert werden. Im Abschnitt \ref{konzept:loesung1} wurde erwähnt, dass reduktion von Ringing Artefakte ein aktives Forschungsfeld der Bildverarbeitung ist. Es existieren Post-Processing Filter, welche Ringing Artefakte vermindern. Eine Möglichkeit die Kompression zu verbessern ist es, einen Post-Processing Filter für wissenschaftliche Daten zu entwickeln.
\pagebreak
\subsection{Lösungsansatz: Prediktive Kodierung}
Wie daten verloren gehen, warum 

\subsubsection{Variante: einfaches Subsampling}
PCA + PCA erlaubt es, 16 Bit zu speichern.
Für den Test dieser Variante wurde den Einfluss von vier Prediktoren überprüft:
\begin{itemize}
\item Konstanter Prediktor: Nimmt an, dass der nächste Wert im Signal gleich dem letzten Wert ist.
\item Linearer Prediktor: Nimmt an, dass der nächste Wert auf der Gerade z
\item Linearer Prediktor mit Moving Average: 
\item Adaptiver Linearer Prediktor mit Moving Average:
\end{itemize}
Wer die beste vorhersage machen kann
\begin{figure}[!htbp]
	\center
	\includegraphics[width=1\textwidth,keepaspectratio]{./pictures/resultate/loesung2/variante0/resultate.png}
	\caption{Kompressionsraten der vier Prediktoren im Vergleich zum Ist-Zustand}
	\label{resultate:loesung2:simple:resultate}
\end{figure}
Im Diagramm der Abbildung \ref{resultate:loesung2:simple:resultate} sind die Kompressionsraten der jeweiligen Prediktoren dargestellt. Ein Diagramm mit der PSNR-HVS-M wurde nicht erstellt. Sie ist für alle Prediktoren gleich und liegt bei $140.7$ dB. Unerwartet ist, dass der Konstante Prediktor mit $255$ Bytes pro Feldlinie die beste Kompression erreichte, obwohl die Daten nicht zuverlässig vorhersagen kann. Im Vergleich mit dem Moving Average Prediktor sind die Fehler der Vorhersagen bis zu $5$ Mal grösser, verbrauchen aber $40$ Bytes weniger um eine Feldlinie abzuspeichern. Der Fehler bleibt jedoch Konstant. Eine Möglichkeit ist, dass die Rar Kodierung sich wiederholende Muster findet.\\
Eine mögliche Optimierung ist die Adaptive Byte Kodierung der DCT-Variante, beschrieben im Abschnitt \ref{konzept:loesung1:kodierung}. Das Diagramm der Abbildung \ref{resultate:loesung2:simple:resultate_byte} zeigt die Resultate mit der Byte Kodierung. Der Konstante Prediktor verbraucht mit der Adaptiven Kodierung mehr Speicherplatz. Die Kompressionsrate der anderen Prediktoren wird durch die Adaptive Kodierung deutlich verbessert. Der Lineare Prediktor erreicht mit $214$ Bytes pro Feldlinie die beste Kompression. Das bedeutet, dass die Fehler des Konstanten Prediktors grösser sind, als die der anderen Prediktoren. Es bestätigt die Vermutung, dass die anderen Prediktoren die Daten besser vorhersagen können. Die Kompressionsrate des Konstanten Prediktors ist auf die Rar Kodierung zurückzuführen, welche in den Prediktor-Fehler Muster erkennen und effizient kodieren kann.\\ 
\begin{figure}[!htbp]
	\center
	\includegraphics[width=1\textwidth,keepaspectratio]{./pictures/resultate/loesung2/variante0/resultate_byte.png}
	\caption{Artefakte der abschliessenden Variante. Links ohne Glättung, rechts mit Glättung}
	\label{resultate:loesung2:simple:resultate_byte}
\end{figure}
Der Linare Prediktor kann eine Feldinie mit etwa $214$ Bytes darstellen, was eine Kompressionsrate von $4$ ergibt. Auf kosten der Qualität wird versucht eine bessere kompressionsrate zu erreichen.

\subsubsection{Variante: Adaptiven Subsampling}
Gute Möglichkeit viele informationen zu verlieren.
Figure
schwieriger zu quantisieren, da weniger Stetig. Kurven der PCA. Möglichkeit eine Variante Prediktive kodierung für Kurven, wird aber einiges komplexer.
Einfacher wenn Kodierung für
Figure
Sphärisches Koordinatensystem.Einfacher abzuspeichern
POW
Aber Ebenfalls schwer vorherzusagen

\subsubsection{Wavelet Prediktive Kodierung}
beschrieben im Abschnitt \ref{konzept:prediktiv}
figure
figure psnr-hvs-m
