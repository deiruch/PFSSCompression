\section{Fazit}
Die Ist-Kompression benötigt durchschnittlich 1 Megabyte an Daten pro Simulation der Feldlinien. Die Zwischenspeicherung von $1000$ Simulationen verbraucht ein Gigabyte an Arbeitsspeicher. Da der Arbeitsspeicher ebenfalls für die Zwischenspeicherung weiterer Daten benötigt wird, ist die Datenmenge nicht vertretbar. Um das Zwischenspeichern der Simulationen zu ermöglichen, ist eine Kompressionsrate von Faktor $8-10$ notwendig. Die entwickelten Kompressionsverfahren Adaptives Subsampling, DCT Kompression und Prädiktive Kodierung erreichten eine Kompressionsrate von $11.6$, $14.1$ und $13.6$. Das Zwischenspeichern von $1000$ Simulationen benötigt zwischen $70$ und $85$ Megabyte Arbeitsspeicher. Die Zwischenspeicherung ist mit den drei entwickelten Kompressionen realisierbar.

Ein Forschungsziel ist, unter welchen Bedingungen ein Streaming der Simulationen. Wird angenommen, dass für die Feldlinien $5$ Megabit Bandbreite zur Verfügung stehen, können im Ist-Zustand $0.6$ Simulationen in der Sekunde heruntergeladen werden. Der JHelioviewers benötigt im Allgemeinen $1$ bis maximal $10$ Simulationen in der Sekunde für die Visualisierung. Mit derselben Bandbreite erreichen die entwickelten Kompressionen durchschnittlich $7$, $9$ und $8$ Simulationen, welche pro Sekunde heruntergeladen werden können. Der Maximalfall von $10$ Simulationen in der Sekunde benötigt eine höhere Bandbreite oder eine höhere Kompression. Für den allgemeinen Fall ist mit den entwickelten Kompressionen und einer modernen Internetverbindung das Streaming möglich. 

Das Auftreten von Ringing oder Ringing ähnlichen Artefakten ist momentan der limitierende Faktor für die Kompression: Im JHelioviewer sind auch leicht ausgeprägte Ringing Artefakte störend, da sie durch das Zoom Feature entdeckt werden können. Wenn für das Fernziel, eine flüssige Animation der Feldlinien, die übertragenen Simulationen pro Sekunde erhöht werden, muss auf den artefaktfreien Zoom verzichtet werden. Durch diesen Verzicht kann eine höhere Kompressionsrate erreicht werden. Das DCT Verfahren eignet sich für eine möglichst hohe Kompressionsrate: Bei steigender Kompressionsrate sinkt die Qualität der Daten langsamer als bei den anderen Kompressionsverfahren.

Das Verfahren des Adaptiven Subsamplings ist einfach umzusetzen und beinhaltet minimale Artefakte. Die Kompressionsrate wurde erreicht, indem nur Daten übertragen werden, welche in der Visualisierung angezeigt werden. Wenn in Zukunft mehr Daten visualisiert werden, muss der JHelioviewer entweder eine Interpolation durchführen oder das Adaptive Subsampling muss mehr Daten übertragen, wobei die Kompressionsrate sinken wird.

Die Kompression mit der Diskreten Kosinus Transformation erreichte die höchste Kompressionsrate aller Verfahren. Die Kompression fügt Ringing Artefakte ein, welche in der Visualisierung stören. Die Artefakte können durch eine Glättung behoben werden. Die Umsetzung der Dekompression gestaltet sich Komplex: Eine naive Implementation der inversen Kosinus Transformation führt zu einer Laufzeit, die um Faktor $100$ Langsamer ist als eine naive Implementation der anderen Kompressionsverfahren. Forschungsinstitutionen wie Beispielsweise IRAP\cite{website:irap} sind an Feldliniensimulationen interessiert. Es ist von Vorteil wenn die Dekompression mit moderaten Programmierkenntnissen umgesetzt werden kann. Die Dekompression dieses Verfahrens beinhaltet die grösste Komplexität.

Die Prädiktive Kodierung erreichte eine vergleichbare Kompressionsrate wie die des DCT Verfahrens. Die Artefakte sind weniger Ausgeprägt und erst bei hohen Zoomstufen erkennbar. Mit einer leichten Glättung sind die Artefakte in der Visualisierung nicht mehr zu erkennen. Die Implementation Dekompression gestaltet sich im Vergleich zum DCT Verfahren simpel. Es wird keine aufwändige Rückwärtstransformationen benötigt, was sich auf die Laufzeit und die Komplexität der Implementation auswirkt. Dieser Ansatz ist also auch mit moderaten Programmierkenntnissen umsetzbar. Das Verfahren der Prädiktiven Kodierung wurde als finale Lösung ausgewählt. Es ist ein Kompromiss zwischen Kompression, Artefakte und Komplexität der Dekompression.

Das Kompressionsverfahren der Prädiktiven Kodierung kann auch auf anderen Anwendungsfälle benutzt werden wie Kompression von Feldlinien von Spulen, Flugbahnen der Teilchenphysik verwendet werden. Anders als bei der DCT ist es für die Kompression nicht relevant, ob die Daten kontinuierliche Funktionen darstellen. Für einen anderen Anwendungsfall muss die Quantisierung angepasst werden um für diesen Fall den Kompromiss zwischen Artefaktbildung zu Kompressionsrate zu finden.

Für weitere Forschungen sind Kompressionsverfahren interessant, welche kaum oder keine Ringing Artefakte einfügen wie die Wavelet Transformation, Compressive Sensing und Curve Fitting. Die Artefakte der Wavelet Transformation kann durch Auswahl des Wavelets beeinflusst werden. Compressive Sensing kann auf die zu komprimierenden Daten optimiert werden. Es kann die Eigenschaft genutzt werden, dass die Feldlinien sich hauptsälchlich in Skalierung, Rotation und Verschiebung unterscheiden. Curve Fitting ist ebenfalls ein mögliches Verfahren. Die Artefakte einer Bildkompression mit Curve Fitting sind Rauschunterdrückung und allgemeine Schärfung des Bildes. Ähnliche Kompressionsartefakte können entstehen, wenn man es für die Feldliniendaten verwendet.

