\section{Fazit}
Die Ist-Kompression benötigt durchschnittlich 1 Megabyte an Daten pro Simulation der Feldlinien. Das Caching von $1000$ Simulationen verbraucht 1 Gigabyte an Arbeitsspeicher. Da der Arbeitsspeicher ebenfalls für die Zwischenspeicherung weiterer Daten benötigt wird, ist diese Datenmenge nicht vertretbar. Um das Zwischenspeichern der Simulationen zu ermöglichen, ist eine Kompressionsrate von Faktor $8-10$ notwendig. Die entwickelten Kompressionsverfahren Adaptives Subsampling, DCT Kompression und Prädiktive Kodierung erreichten eine Kompressionsrate von $11.6$, $14.1$ und $13.6$. Das Zwischenspeichern von $1000$ Simulationen benötigt zwischen $70$ und $85$ Megabyte Arbeitsspeicher. Somit ist das Caching mit allen entwickelten Kompressionsverfahren performant.

Weiterhin wurde das Streaming der Feldlinien-Simulationen erforscht: Mit der selben Bandbreite können im Ist-Zustand $0.6$ Simulationen in der Sekunde heruntergeladen werden. Der JHelioviewers benötigt bis zu $10$ Simulationen in der Sekunde für die Visualisierung. Die entwickelten Kompressionsverfahren erreichen einen Durchsatz von durchschnittlich $8$ Simulationen in der Sekunde. Für den allgemeinen Anwendungsfall ist mit den entwickelten Kompressionen und einer modernen Internetverbindung das Streaming möglich.

Das Kompressionsverfahren mittels Prädiktiver Kodierung wurde als finale Kompression ausgewählt. Das Verfahren erreichte die höchste Qualität zur zweitbesten Kompressionsrate. Die Artefakte der Kompression wirken sich meistens als Verschiebungen aus. In der Visualisierung sind solche Artefakte für das menschliche Auge schwieriger zu entdecken, als die Artefakte der DCT Kompression. Gegenüber der DCT Kompression ist die Prädiktive Kodierung einfach zu implementieren: Es werden keine komplexen Transformationen verwendet, eine Implementation der Dekompression ist weniger anspruchsvoll. Die Feldliniendaten sind öffentlich zugänglich, eine einfache Implementation vereinfacht Dritten wie Beispielsweise IRAP \cite{website:irap} den Zugriff. Das Verfahren mittels Adaptives Subsampling ist ähnlich wie die Prädiktive Kodierung einfach zu implementieren und die Kompressionsartefakte sind vom menschlichen Auge schwerer zu erkennen, als die der DCT Kompression. Die Prädiktive Kodierung erreichte die höhere Kompression zu einer höheren Qualität als das Verfahren des Adaptiven Subsamplings. Das Verfahren der Prädiktiven Kompression ist ein Kompromiss zwischen Kompressionsrate, Qualität und Komplexität der Implementation.

Das Verfahren mittels Prädiktiven Kodierung kann für anderen Anwendungsfälle eingesetzt werden wie Kompression von Feldlinien von Spulen oder Flugbahnen der Teilchenphysik. Das entwickelte Dateiformat ist auf kein Koordinatensystem oder maximale Genauigkeit begrenzt. Für eine optimale Kompression muss die Quantisierung auf den jeweiligen Anwendugsfall angepasst werden.

Für weitere Forschungen sind die Verfahren Wavelet-Transformation, Compressive Sensing und Curve Fitting interessant. Die Wavelet-Transformation ist stark mit der Kosinus-Transformation verwandt. In der Wavelet-Transformation hat das Potential eine ähnliche Kompressionsrate wie die DCT Kompression zu erreichen zu einer höheren Qualität.\\
Compressive Sensing repräsentiert ein Signal durch eine minimale Anzahl an Funktionen (sparse representation), welche in einem Dictionary definiert sind. Die Feldlinien unterscheiden sich am stärsten in Rotation, Verschiebung und Skalierung. Die Form der Feldlinien sind aber meist ähnlich. Mit einem geeigneten Dictionary können die Feldlinien durch wenige Funktionen im Dictionary dargestellt werden und das Verfahren erreicht eine hohe Kompression zu einer hohen Qualität. Es existieren Algorithmen für die Berechnung der sparse representation und für das Erstellen des Dictionary's. Die Algorithmen sind aber im Vergleich zu Kosinus- oder Wavelet Transformation komplex zu implementieren. Für eine Datenkompression müssen Detailprobleme, wie Beispielsweise das Dictionary der Dekompression zur Verfügung gestellt wird, gelöst werden. Das sparse representation ist Rechenintensiv und die Laufzeit der Kompression wird deutlich höher sein als bei Verfahren mittels Prädiktiver Kodierung oder DCT.\\
Curve Fitting wird für Signalinterpolation oder Rauschunterdrückung verwendet. Die Artefakte einer Kompression mittels Curve Fitting drücken sich als Glättung aus und sind für das menschliche Auge schwieriger zu entdecken. In der Datenkompression ist Curve Fitting kein etabliertes Verfahren. Es existieren hauptsächlich Machbarkeitsnachweise, aber ein Kompressionsstandard mittels Curve Fitting ist nicht bekannt.

Ebenfalls stehem Programmbibliotheken, welche ein Curve-Fitting anbieten, nur beschränkt zur Verfügung.vergleichsweise unerforscht. Es existieren Machbarkeitsnachweise. Es fehlen ebenfalls performante oder feature rich implementationen von 

\pagebreak

 Eine Datenkompression mittels Curve Fitting ist möglich, aber kein etabliertes Verfahren und deshalb vergleichsweise unerforscht.

dliche Wavelet-Funktionen verwendet werden, welche die Artefaktbildung beeinflussen. Die Wavelet-Transformation
Für weitere Forschungen sind Kompressionsverfahren interessant, welche kaum oder keine Ringing-Artefakte einfügen wie die Wavelet Transformation, Compressive Sensing und Curve Fitting. Die Artefakte der Wavelet Transformation kann durch Auswahl des Wavelets beeinflusst werden. Compressive Sensing kann auf die zu komprimierenden Daten optimiert werden. Es kann die Eigenschaft genutzt werden, dass die Feldlinien sich hauptsächlich in Skalierung, Rotation und Verschiebung unterscheiden. Curve Fitting ist ebenfalls ein mögliches Verfahren. Die Artefakte einer Bildkompression mit Curve Fitting sind Rauschunterdrückung und Kantenschärfung des Bildes. Ähnliche Kompressionsartefakte können entstehen, wenn das Verfahren Feldliniendaten angewendet wird.






Die Prädiktive Kodierung erreichte die höchste Kompressionsrate zu minimalen Artefakten. Die Artefakte sind weniger Ausgeprägt und erst bei hohen Zoomstufen erkennbar. Mit einer leichten Glättung sind die Artefakte in der Visualisierung nicht mehr zu erkennen. Das Verfahren verwendet keine komplexen Transformationen: Wenn Institutionen wie Beispielsweise IRAP \cite{website:irap} eine Dekompression entwickelt, sind für die Umsetzung weniger Programmierkenntnisse notwendig als beim DCT-Verfahren. Das Verfahren der Prädiktiven Kodierung wurde als finale Lösung ausgewählt. Es ist ein Kompromiss zwischen Kompression, Artefaktbildung und Komplexität der Implementation.

Das Verfahren des Adaptiven Subsamplings ist einfach umzusetzen und beinhaltet minimale Artefakte. Es werden ausschliesslich die Daten übertragen, welche der JHelioviewer für die Visualisierung benötigt. Die Datenmenge ist begrenzt durch die Leistung der Grafikkarte. Wenn in Zukunft die zu visualisierenden Datenmenge erhöht wird, sinkt die Kompressionsrate des Verfahrens. Das Problem weisen die Kompressionsverfahren DCT und Prädiktive Kodierung nicht auf, da mehr Daten übertragen werden, als die Visualisierung benötigt.

Die Kompression mit der Diskreten Kosinus Transformation erreichte die höchste Kompressionsrate aller Verfahren. Die Kompression fügt Ringing-Artefakte ein, welche in der Visualisierung stören. Die Artefakte können durch eine Glättung behoben werden. Die Umsetzung der Dekompression gestaltet sich Komplex: Eine naive Implementation der inversen Kosinus Transformation führt zu einer Laufzeit, die um Faktor $100$ langsamer ist als eine naive Implementation der anderen Kompressionsverfahren. Forschungsinstitutionen wie Beispielsweise IRAP \cite{website:irap} sind an Feldliniensimulationen interessiert. Es ist von Vorteil wenn die Dekompression mit moderaten Programmierkenntnissen umgesetzt werden kann. Die Dekompression dieses Verfahrens beinhaltet die grösste Komplexität.